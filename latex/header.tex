% Set up the fonts
\usepackage[urw-palatino]{mathdesign}
\usepackage[T1]{fontenc}


% Add accessibility support from http://www.richschwinn.com/accessibility
\RequirePackage{accsupp}
\RequirePackage{pdfcomment}
\newcommand{\AccTool}[2]{\BeginAccSupp{method=pdfstringdef,unicode,Alt={{#1}}}\pdftooltip{{#2}}{{#1}}\EndAccSupp{}}

% Set the language for 508
\hypersetup{
  pdftitle = {title},
  pdflang = en-US}

% Set up the headers and footers
\usepackage{graphicx}
\usepackage{fancyhdr}
\usepackage{ifthen}
\usepackage{everypage}
\usepackage{float}
\usepackage{subfig}

% Avoid struggling over figure and table float in Rmarkdown
\let\origfigure\figure
\let\endorigfigure\endfigure
\renewenvironment{figure}[1][2] {
    \expandafter\origfigure\expandafter[H]
} {
    \endorigfigure
}

\let\origtable\table
\let\endorigtable\endtable
\renewenvironment{table}[1][2] {
    \expandafter\origtable\expandafter[H]
} {
    \endorigtable
}

% First page has the large title and NOAA logo
\pagestyle{fancy}
\fancyhf{}
\setlength\headheight{40pt}
\fancyheadoffset[L]{0.5cm}
\cfoot{\thepage}

\fancyheadinit{%
   \ifthenelse{\value{page}=1}%
      {\fancyhead[R]{\textsf{\emph{7 March 2025}}}
       \fancyhead[L]{\textsf{\LARGE Mid-Atlantic EAFM Risk Assessment: 2025 Update}}
      }%
      {\fancyhead[R]{}
       \fancyhead[L]{\textsf{\emph{Risk Assessment Update 2025}}}
      }
}

\renewcommand{\headrulewidth}{0.4pt}
\renewcommand{\footrulewidth}{0pt}

% Make caption fonts a bit smaller
\usepackage[font={small}]{caption}


% Change section labels to san serif
\usepackage{sectsty}
\allsectionsfont{\normalfont\sffamily\bfseries}